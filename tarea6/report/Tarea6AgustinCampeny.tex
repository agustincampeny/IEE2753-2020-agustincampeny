%! TEX program = xelatex
% Paquetes:
\documentclass[letterpaper, 12pt]{article}
\usepackage[spanish]{babel} %%Paquete español para mac
\usepackage{graphicx} %% Para incluir figuras
\usepackage{xcolor}
\usepackage{ifpdf}
\usepackage{circledsteps}
\DeclareGraphicsExtensions{.pdf}
\usepackage[margin=1in]{geometry}
\setcounter{totalnumber}{5}
\renewcommand{\textfraction}{0.1}
\usepackage[cmex10]{amsmath}
\usepackage{amssymb}
\usepackage{float}
\usepackage{cite}
\bibliographystyle{unsrt}
%\decimalpoint
\usepackage{url}
\usepackage{hyperref}
\hypersetup{colorlinks=false,bookmarksopen=true,linkbordercolor={1 1 1}}
\usepackage{mathtools}
\usepackage{chngcntr}
\usepackage{enumitem}
\providecommand{\e}[1]{\ensuremath{\times 10^{#1}}}
\usepackage[parfill]{parskip} % Líneas en lugar de indentación
\usepackage{fancyhdr}
\usepackage{booktabs}
\usepackage{cleveref}
\usepackage{verbatimbox}
\crefformat{footnote}{#2\footnotemark[#1]#3}

\usepackage[squaren]{SIunits} %esto me da nombres de unidades y prefijos
\usepackage{sistyle}% Esto es adecuado para escribir unidades.

\newcommand{\alumno}{Agustín Campeny}
\lhead{\nouppercase{\leftmark}}
\rhead{Tarea 6 - \alumno}
\pagestyle{fancy}
%\usepackage{scrextend}
\numberwithin{equation}{section}

\setlength{\tabcolsep}{6pt} % General space between cols (6pt standard)
\renewcommand{\arraystretch}{0.8} % General space between rows (1 standard)
\begin{document}
\thispagestyle{empty}
%%%%%%%%%%%%%%%%%%%%%%%%%%
%%%%%%%%% ENCABEZADO %%%%%%%%%
%%%%%%%%%%%%%%%%%%%%%%%%%%
\vspace*{-1cm}
\includegraphics[width=2cm]{logo.pdf}
\vspace*{-2.2cm}

\hspace*{2cm}
 \begin{tabular}{l}
  {\ \textsc{\raggedright \footnotesize Pontificia Universidad Católica de Chile}}\\
  {\ \textsc{\raggedright \footnotesize Escuela de Ingeniería}}\\
  {\ \textsc{\raggedright \footnotesize Departamento de Ingeniería Eléctrica}}\\
  {\ \textsc{\raggedright \footnotesize IEE2753 - Diseño de Circuitos Integrados Digitales}}\\
  {\  }\\
 \end{tabular}
 \hfill
\vspace*{-0.2cm}
\begin{center}
  {\Large\bf Tarea 6}\\
\vspace*{2mm}
{\today}\\
\vspace*{2mm}
{\footnotesize \alumno}\\
\vspace*{6mm}
\end{center}
\hrule\vspace*{2pt}\hrule
%%%%%%%%%%%%%%%%%%%%%%%%%%
%%%%%%%%% ENCABEZADO %%%%%
%%%%%%%%%%%%%%%%%%%%%%%%%%

\section{Factor de actividad}

\subsection{}

Para este ejercicio se calcula la probabilidad para cada nodo, y luego su factor de actividad, definido como \(\alpha = P_i\overline{P_i}\).

\begin{itemize}
  \item \(n_0\): Como la probabilidad de todas las entradas del circuito son de 0.5, y la probabilidad de una puerta NAND2 es de \(P_{\text{NAND2}} = 1-P_AP_B\), la probabilidad \(P_{n_0} = 0.75\), y el factor de actividad \(\alpha_{n_0} = 0.1875\).

  \item \(n_1\): El inversor solo invierte la probabilidad del nodo de entrada, pero no modifica el factor de actividad, por lo tanto \(P_{n_1} = 0.25\) y \(\alpha_{n_1} = 0.1875\).

  \item \(n_2\): Se utiliza la probabilidad de \(n_1\) y de la nueva entrada. El valor \(P_{n_2} = 0.875\) y \(\alpha_{n_2} = 0.109375\).

  \item \(n_3\): Este nodo es idéntico a \(n_2\), por lo tanto \(P_{n_3} = 0.875\) y \(\alpha_{n_3} = 0.109375\).
\end{itemize}

\subsection{}

Despreciando la potencia interna, se define la potencia dinámica como la suma de las potencias de switching en las entradas de cada compuerta y en la capacitancia de carga. De esta forma:

\begin{equation}
  P_{\text{dynamic}} = \left( 0.5 C_{in} + 0.1875 C_{1} + 0.875 C_2 + 0.21875 C_{Load}\right) V_{DD}^2 f
\end{equation}

\section{Low Power Placement}

Se quiere minimizar el factor de actividad en los nodos con capacitores. Para esto primero se determina el valor del factor de actividad con respecto a las entradas:

\begin{align}
  \alpha_{1} &= (1-\overline{P_{in1}}\overline{P_{in2}})\cdot \overline{P_{in1}}\overline{P_{in2}} \\
  \alpha_{2} &= (1-\overline{P_{int}}\overline{P_{in3}})\cdot \overline{P_{int}}\overline{P_{in3}} \\
\end{align}

Como el nodo de salida es una función de las tres entradas, su factor de actividad es el mismo sin importar el orden de estas. Se busca entonces minimizar \(\alpha_1\).

El orden que minimiza este valor es \(P_{in1} = 0.1\) y \(P_{in2} = 0.2\), correspondiendo a un factor de actividad \(\alpha_{1}^{\text{min}} = 0.0196 \).

El orden que maximiza este valor es \(P_{in1} = 0.2\) y \(P_{in2} = 0.5\), correspondiendo a un factor de actividad \(\alpha_{1}^{\text{min}} = 0.09 \).

\end{document}
