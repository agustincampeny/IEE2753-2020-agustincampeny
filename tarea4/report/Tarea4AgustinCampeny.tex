%! TEX program = xelatex
% Paquetes:
\documentclass[letterpaper, 12pt]{article}
\usepackage[spanish]{babel} %%Paquete español para mac
\usepackage{graphicx} %% Para incluir figuras
\usepackage{xcolor}
\usepackage{ifpdf}
\usepackage{circledsteps}
\DeclareGraphicsExtensions{.pdf}
\usepackage[margin=1in]{geometry}
\setcounter{totalnumber}{5}
\renewcommand{\textfraction}{0.1}
\usepackage[cmex10]{amsmath}
\usepackage{amssymb}
\usepackage{float}
\usepackage{cite}
\bibliographystyle{unsrt}
%\decimalpoint
\usepackage{url}
\usepackage{hyperref}
\hypersetup{colorlinks=false,bookmarksopen=true,linkbordercolor={1 1 1}}
%\usepackage{epstopdf}
\usepackage{mathtools}
\usepackage{chngcntr}
\usepackage{enumitem}
\providecommand{\e}[1]{\ensuremath{\times 10^{#1}}}
\usepackage[parfill]{parskip} % Líneas en lugar de indentación
\usepackage{fancyhdr}
\usepackage{booktabs}
\usepackage{cleveref}
\usepackage{verbatimbox}
\crefformat{footnote}{#2\footnotemark[#1]#3}

\usepackage[squaren]{SIunits} %esto me da nombres de unidades y prefijos
\usepackage{sistyle}% Esto es adecuado para escribir unidades.

\newcommand{\alumno}{Agustín Campeny}
\lhead{\nouppercase{\leftmark}}
\rhead{Tarea 3 - \alumno}
\pagestyle{fancy}
%\usepackage{scrextend}
\numberwithin{equation}{section}

\setlength{\tabcolsep}{6pt} % General space between cols (6pt standard)
\renewcommand{\arraystretch}{0.8} % General space between rows (1 standard)
\begin{document}
\thispagestyle{empty}
%%%%%%%%%%%%%%%%%%%%%%%%%%
%%%%%%%%% ENCABEZADO %%%%%%%%%
%%%%%%%%%%%%%%%%%%%%%%%%%%
\vspace*{-1cm}
\includegraphics[width=2cm]{logo.pdf}
\vspace*{-2.2cm}

\hspace*{2cm}
 \begin{tabular}{l}
  {\ \textsc{\raggedright \footnotesize Pontificia Universidad Católica de Chile}}\\
  {\ \textsc{\raggedright \footnotesize Escuela de Ingeniería}}\\
  {\ \textsc{\raggedright \footnotesize Departamento de Ingeniería Eléctrica}}\\
  {\ \textsc{\raggedright \footnotesize IEE2753 - Diseño de Circuitos Integrados Digitales}}\\
  {\  }\\
 \end{tabular}
 \hfill
\vspace*{-0.2cm}
\begin{center}
  {\Large\bf Tarea 3}\\
\vspace*{2mm}
{\today}\\
\vspace*{2mm}
{\footnotesize \alumno}\\
\vspace*{6mm}
\end{center}
\hrule\vspace*{2pt}\hrule
%%%%%%%%%%%%%%%%%%%%%%%%%%
%%%%%%%%% ENCABEZADO %%%%%
%%%%%%%%%%%%%%%%%%%%%%%%%%

\section{Intento de buffer}











\end{document}
